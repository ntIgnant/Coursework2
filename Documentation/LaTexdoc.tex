\documentclass{article}
\usepackage{natbib}
\usepackage{amsmath}

\title{LaTeX Documentation}
\author{Jiahao, Ignacio, Zibo}
\date{\today}

\setlength{\parindent}{0pt}
\setlength{\parskip}{1em}

\begin{document}

\maketitle
\newpage

\section{Introduction}

This document contains the documentation of the second coursework of the ecosystems class. For this project, a web page was made to test the proficiency in HTML, CSS, JavaScript, and XML mainly. Also, Git was used for the collaborative work, where the whole project was developed in a repository on the GitHub platform.

To create this webpage, Git and Github were used a lot among the members of the group, this to have a professional workflow to share the different versions and changes of the project among the members.

In this document, the methods used for the accomplishment of the project will be explained, as well as the difficulties and findings that were obtained along the development of the project.

\section{Methods}

As previously mentioned, different languages were used for this project.

\begin{itemize}
  \item \textbf{Git/Github} \\
  For this project, a lot of group work was involved, so Git technology was used. All the work, from start to end, was done in a repository in Github, where all the members of the group collaborated in the development of the webpage, each change to the code was commented and uploaded to Github so that the collaborators can see it and update it with changes.

  \item \textbf{HTML/CSS} \\
  A big part of this projects contains html and css code. HTML to give the structure and content of the page, like images and text. CSS to give the style, colors, text font, and also a big part of the structure of the web page with flexbox, this to make the page responsive depending on the device, no matter if it is being displayed from a mobile device, the structure of the web page still keeps coherence for a good user experience.

  \item \textbf{XML/JavaScript} \\
  For the interaction of the web page with the user, JavaScript was used, a button was made so that the user can copy the movie data in XML format. Additionally, the data from the movieInfo.xml file can be viewed in the DOM.
\end{itemize}

\section{Findings}

1. \textbf{Github}

Sharing a common repository of three team member on Github, a clear project structure helps management, collaboration, and mutual understanding of different ideas and code writing history among team members. Store HTML files, CSS files, JAVASCRIPT files, XML files, image resources and documents in different folders, such as index.html, Styles.css, JS.script, movieInfo.xml.

2. \textbf{HTML/CSS}

Using HTML tags (such as <header>, <main>, <section>, <nav>, etc.) can improve the readability and maintainability of the code.
The importance of using flexible layout (Flexbox) is that it enables the responsive pages we create to have good display effects on different devices.

3. \textbf{JavaScript}

Use JavaScript to achieve interactive effects, such as options and drop-down pages at the top of the page, so as to enhance the interactivity of the page
Review: There are no large number of interactive options and buttons.

4. \textbf{XML}

XML is an excellent data organization method that can be used to organize and transmit movie information. The information of each recommended movie can be represented by an XML node, including title, director, main actor, release year, poster URL.

\section{Conclusion}

Throughout the development of this movie recommendation website project, our team has gained valuable skills and experience in various aspects of web development and collaborative work. It not only allowed us to integrate various web technologies such as HTML, CSS, JavaScript, and XML, but also made us aware of potential improvements in practice, such as more advanced GitHub usage techniques, more complex CSS styling, and richer JavaScript interaction features. Overall, this experience consolidated our theoretical knowledge, allowed us to face the challenges of practical development, and built a solid foundation for our future development in web design and team projects.

\end{document}
