\documentclass{article}
\usepackage{natbib}
\usepackage{amsmath}
\title{LaTeX Documentation}
\author{Jiahao, Ignacio, Zibo}
\date{\today}

\setlength{\parindent}{0pt}
\setlength{\parskip}{1em}

\begin{document}

\maketitle

\section{Introduction}

This document contains the documentation of the second coursework of the ecosystems class. For this project, a web page was made to test the proficiency in HTML, CSS, JavaScript, and XML mainly. Also, Git was used for the collaborative work, where the whole project was developed in a repository on the GitHub platform.

To create this webpage, Git and Github were used a lot among the members of the group, this to have a professional workflow to share the different versions and changes of the project among the members.

In this document, the methods used for the accomplishment of the project will be explained, as well as the difficulties and findings that were obtained along the development of the project.


\section{Methods}

As previously mentioned, different languages were used for this project.

\begin{itemize}
  \item \textbf{Git/Github} \\
  For this project, a lot of group work was involved, so Git technology was used. All the work, from start to end, was done in a repository in Github, where all the members of the group collaborated in the development of the webpage, each change to the code was commented and uploaded to Github so that the collaborators can see it and update it with changes.

  \item \textbf{HTML/CSS} \\
  A big part of this projects contains html and css code. HTML to give the structure and content of the page, like images and text. CSS to give the style, colors, text font, and also a big part of the structure of the web page with flexbox, this to make the page responsive depending on the device, no matter if it is being displayed from a mobile device, the structure of the web page still keeps coherence for a good user experience.

  \item \textbf{XML/JavaScript} \\
  For the interaction of the web page with the user, JavaScript was used, a button was made so that the user can copy the movie data in XML format. Additionally, the data from the movieInfo.xml file can be viewed in the DOM.

\end{itemize}
\section{Findings}


    
\section{Conclusion}


\bibliographystyle{abbrv}
\bibliography{Reference}


\end{document}